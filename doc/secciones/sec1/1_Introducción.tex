\section{Introducción}

Esta es la primera práctica de la asignatura \textit{Inteligencia de Negocio}, impartida en la Universidad de Granada, en el primer cuatrimestre del curso 2025/2026.

En esta práctica exploraremos algoritmos de \textit{machine learning} conocidos como clasificadores. En particular, nos enfocaremos en dos tipos: clasificadores binarios y multi-clase, ambos diseñados para devolver una única clase como resultado. Estos clasificadores pertenecen a los modelos de aprendizaje supervisado, donde el algoritmo se entrena con datos etiquetados que contienen las respuestas correctas, permitiéndole aprender patrones que luego aplicará en nuevos datos para predecir la clase correspondiente.

Los datos que emplearemos serán tabulares, es decir, estarán organizados en filas y columnas, donde cada fila representa una instancia o muestra y cada columna representa una característica o atributo de esas instancias. Las características pueden ser de diferentes tipos, como numéricas, categóricas o booleanas.

Además, realizaremos un análisis exploratorio de datos (\textit{Exploratory Data Analysis}, EDA) previo sobre cada conjunto de datos, utilizando tres datasets para abordar problemas aplicados y reales.


\subsection{Software.}

Trabajaremos con KNIME \cite{berthold2009knime}, la plataforma de análisis de datos y \textit{machine learning} que nos permitirá implementar y evaluar algoritmos de clasificación.
KNIME nos proporciona una interfaz visual e intuitiva para trabajar con los datasets.

También usaremos bibliotecas de Python para el análisis de datos como Scikit-Learn \cite{fabian2011scikit} y visualización de datos, como Plotly \cite{plotly} y Matplotlib \cite{hunter2007matplotlib}.

